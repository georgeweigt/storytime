% author uses British spelling "marvellous" and "cancelled"
% change phenomental to phenomenal

\documentclass[12pt]{article}
\title{Heisenberg and the Early Days of Quantum Mechanics}
\author{Felix Bloch}
\date{1976}
\pdfinfoomitdate=1
\pdftrailerid{}
\begin{document}

\maketitle

\noindent
It is appropriate in this year, when we celebrate the 50th anniversary of quantum mechanics, and during which we have been saddened by the death of one of its leading founders, Werner Heisenberg, to reminisce about the formative years of the new mechanics. At the time when the foundations of physics were being replaced with totally new concepts I was a student of physics. I sat in the colloquium audience when Peter Debye made the suggestions to Erwin Schr\"odinger that started him on the study of de Broglie waves and the search for their wave equation. It was from Heisenberg, as his first doctorate student, that I caught the spirit of research, and that I received the encouragement to make my own contributions.

\bigskip
\noindent
{\bf First inklings}

\smallskip
Let me begin by going back to 1924, when I entered the Swiss Federal Institute of Technology in my home town of Zurich. I began as a student of engineering but after a year and good deal of soul searching I decided, against all good sense, to switch over to the ``entirely useless'' field of physics. The E.~T.~H., as it is known from its German name, was an institution of great international repute and in my newly chosen field of studies I
had heard of such famous men as Peter Debye and Hermann Weyl. In fact, the first introductory course of physics I took was taught by Debye and, without knowing much about his scientific work, I realized from the high quality of his lectures at the Institute that here was a great master of his field.

There was a good deal less to be enthusiastic about in the other courses one
could take, and there was nothing like the complete menu that is presented to the students nowadays. Once in a while, a professor would offer a special course on a subject he just happened to be interested in, completely disregarding the tremendous gaps in our knowledge left by this system. Anyway, there was only a handful of us foolish enough to study physics and it was evidently not thought worthwhile to bother much about these ``odd fellows.'' The only thing we could do about it was to go to the library and read some books, although nobody would advise us which ones to choose.

Among the first I hit upon was Arnold Sommerfeld's {\it Atomic Structure and Spectral Lines}, which I found fascinating; the only trouble was that I could not understand most of it because I knew far too little of mechanics and electrodynamics. So at first I had to learn about these subjects from other books, to truly appreciate what Sommerfeld said; but then it conveyed the good feeling that everything about atoms was completely known and understood. The fact that one really could handle only periodic systems and only those that allowed a separation of variables did not seem a great cause for concern. Therefore, when I saw a paper in which somebody tried to squeeze the theory of the Compton Effect into that scheme, I was more impressed than discouraged by the complicated mathematics spent in the effort.

The news that the foundations of a new mechanics had already been laid by Maurice de Broglie and Heisenberg had hardly leaked to Zurich yet and certainly had not penetrated to our lower strata. The first inklings of such a thing came to me in early 1926; I had by then started to attend the physics colloquium regularly, although most of what I heard there was far above my head. The colloquium, run
with firm authority by Debye, might have had an audience of as much as a couple of dozen---on a good day.

Physics was also taught at the University of Zurich by a smaller and rather less illustrious faculty than that at the E. T. H. Theory there was in the hands of a certain Austrian of the name of Schr\"odinger, and the colloquium was alternately held at both institutions. I apologize to my friends who already have heard from me what I am going to tell you now. My account may not conform to the strictest standards of history, which accord validity only to written documents, nor will I be able to render the exact words I heard on those occasions, but I can vouchsafe that, in content, I shall report the truth and only the truth.

\bigskip
\noindent
{\bf A wave equation is found}

\smallskip
Once at the end of a colloquium I heard Debye saying something like: ``Schr\"odinger, you are not working right now on very important problems anyway. Why don't you tell us some time about that thesis of de Broglie, which seems to have attracted some attention.''

So, in one of the next colloquia, Schr\"odinger gave a beautifully clear account of how de Broglie associated a wave with a particle and how he could obtain the quantization rules of Niels Bohr and Sommerfeld by demanding that an integer number of waves should be fitted along a stationary orbit. When he had finished, Debye casually remarked that he thought this way of talking was rather childish. As a student of Sommerfeld he had learned that, to deal properly with waves, one had to have a wave equation. It sounded quite trivial and did not seem to make a great impression, but Schr\"odinger evidently thought a bit more about the idea afterwards.

Just a few weeks later he gave another
talk in the colloquium which he started by saying: ``My colleague Debye suggested that one should have a wave equation; well, I have found one!''

And then he told us essentially what he was about to publish under the title ``Quantization as Eigenvalue Problem'' as a first paper of a series in the {\it Annalen der Physik}. I was still too green to really appreciate the significance of this talk, but from the general reaction of the audience I realized that something rather important had happened, and I need not tell you what the name of Schr\"odinger has meant from then on. Many years later, I reminded Debye of his remark about the wave equation; interestingly enough he claimed that he had forgotten about it and I am not quite sure whether this was not the subconscious suppression of his regret that he had not done it himself. In any event, he turned to me with a broad smile and said: ``Well, wasn't I right?''

Of course, there was afterwards a lot of talk among the physicists of Zurich, including even the students, about that mysterious ``psi'' of Schr\"odinger. In the summer of 1926, a fine little conference was held there and at the end everyone
joined a boat trip to dinner in a restaurant on the lake. As a young {\it Prwatdozent}, Erich H\"uckel worked at that time on what is now well known as the Debye-H\"uckel theory of strong electrolytes, and on the occasion he incited and helped us to compose some verses, which did not show too much respect for the great professors. As an example, I want to quote the one on Erwin Schr\"odinger in its original German:

\begin{quote}
\it
``Gar Manches rechnet Erwin schon\\
Mit seiner Wellenfunktion.\\
Nur wissen m\"ocht' man gerne wohl\\
Was man sich dabei vorstell'n soll.''
\end{quote}

\noindent
In free translation:

\begin{quote}
Erwin with his psi can do\\
Calculations quite a few.\\
But one thing has not been seen:\\
Just what does psi really mean?
\end{quote}

Well, the trouble was that Schr\"odinger did not know it himself. Max Born's interpretation as probability amplitude came only later and, along with no less a company than Max Planck, Albert Einstein and de Broglie, he remained skeptical about it to the end of his life. Much later, I was once in a seminar where someone drew certain quite extended conclusions from the Schr\"odinger equation, and Schr\"odinger expressed his grave doubts that it could be taken that seriously; whereupon Gregor Wentzel, who was also there, said to him: ``Schr\"odinger, it is most fortunate that other people believe more in your equation than you do!''

Schr\"odinger thought for a time that a wave packet would represent the actual shape of an electron, but it naturally bothered him that the thing had a tendency to spread out in time as if the electron would gradually get fatter and fatter.

As I said before, I was too green then to understand these things and still struggled with the older theories. In reading Debye's paper of 1923 on the Compton effect, it occurred to me that, instead of
his assumption of the electron being originally at rest, one should take into account its motion on a stationary orbit in the atom. I thought this was such a good idea that I even had the incredible courage to go to Debye's office and tell it to him. It really wasn't all that wrong but he only said: ``That's no way any more to talk about atoms; you better go and study Schr\"odinger's new wave mechanics.''

Well, you would not disobey the authorities and, of course, he was again quite right. So this is what I did; Schr\"odinger's next papers on wave mechanics appeared shortly, one after the other. I did not learn about the matrix formulation of quantum mechanics by Heisenberg, Born and Pascual Jordan until I read that paper of Schr\"odinger's in which he showed the two formulations to lead to the same results. It did not take me too long to absorb these new methods, and I wish I could confer to the younger physicists who read this article the marvellous feeling we students experienced at that time in the sudden tremendous widening of our horizon. Since we were not burdened with much previous knowledge, the process was quite painless for us, and we were blissfully unaware of the deep underlying change of fundamental concepts that the more experienced older physicists had to struggle with.

Although I had already begun an experiment in spectroscopy, I was now entirely captured by theory and I felt the legal entrance into the guild to be confirmed through my acquaintance with Walter Heitler and Fritz London. They had just obtained their PhD's and had come to Schr\"odinger's Institute, where together they worked on their theory of covalent bonds. I must have met them in a seminar, and it was a great thing for me that they asked me to join them in some of their walks through the forests around Zurich. For us students the professors lived somewhere in the clouds, and that two real theorists at the ripe age of almost 25 should even bother about a greenhorn like me was ample cause for my gratitude to them.

\bigskip
\noindent
{\bf Leipzig}

\smallskip
This great period in Zurich came to a sudden end in the fall of 1927 when some of the most important men there simultaneously succumbed to the pull of the large magnet in the North, represented by the flourishing science in Germany. Weyl had accepted a position in G\"ottingen, Schr\"odinger in Berlin and Debye in Leipzig, and it was clear to me that I had to join the exodus if I did not want my time as a student to drag on much longer. The question was only where to go; I was tempted to follow either London's example and go with Schr\"odinger to Berlin, or Heitler's, and go to G\"ottingen.

Before deciding, however, I went to ask Debye for his opinion, and he advised me to do neither but instead to come to
Leipzig. There I would work with Heisenberg whom he, as the new director of the Institute of Physics of the University, had persuaded to accept the professorship for theoretical physics. Debye's power of persuasion was quite formidable and I could not resist it either, particularly because I had previous evidence of his sound judgment.

So, in October 1927 before the beginning of the winter semester, I left my nice home town for the first time, to arrive on a cold gray morning in that rather ugly city of Leipzig. The little room I found for rent from a family overlooked a railroad yard; the noise and smoke did not help much to cheer me up! As soon as I had completed the simple formality of registering as a student of the University in the center of the city I went to the Physics Institute, which was located near the outskirts.

It was an old building opposite a cemetery on one side and adjoining the garden of a mental institution on the other, but occupied by people who were far from being either dead or crazy. Heisenberg had not arrived yet and the theorist in charge was Wentzel who, a year later, was to become Schr\"odinger's successor in Zurich. I did not find him in his office and was told by an assistant that I could see him in his apartment on the third floor of the building.

It was quite customary at that time for professors to have official living quarters in or adjacent to their institutes; Debye had the Director's villa in a side wing, and for young bachelors like Wentzel and also Heisenberg upon his arrival there were small but comfortable apartments under the roof.

I was not at all sure whether it was really all right to go up there and knock at his door but I dared to do it anyhow, and almost from the moment he opened it I realized that I had come to a new and much warmer academic climate. Used to the great distance that separated the students and professors in freedom-loving Switzerland, I had expected the proverbial discipline of the Germans to call for an even stricter caste system. Instead, Wentzel received me with the informal cordiality of a colleague, which made it almost difficult for me to address him
with the normal {\it ``Herr Professor''} but very easy to show him a little paper I had written before I came to Leipzig.

My paper had been motivated by Schr\"odinger's old dislike of electron wavepackets' disagreeable habit of spreading, and I had had the naive idea that they might be cured from it at least partially by radiation damping. To try it out, I had done a serious calculation for the harmonic oscillator, with the result that a suitable gaussian wavepacket,
without spreading, would perform a nice damped oscillation that led asymptotically to the wavefunction of the ground state. Wentzel made some kind comments but modestly disclaimed sufficient expert knowledge to pass judgment; he said I should ask Heisenberg, who was expected in a few days.

\bigskip
\noindent
{\bf My first paper}

\smallskip
Although his great achievements dated back no more than about two years, Heisenberg was already very famous as the founder of the new form of mechanics, which accounted for quantum phenomena by abandoning such fundamental ideas as motion in an orbit and replacing them by concepts referring to the actual observation of atomic processes. I think I lost my breath for a moment when Wentzel introduced me to this great physicist in the person of a slender young man. Maybe Debye had already mentioned to him that he knew me from Zurich; in any case, as soon as he shook hands and started to talk to me in his simple natural way, I had the feeling that I was ``accepted.''

Just as with Wentzel, there was no indication whatever of a barrier to separate us on the grounds of Heisenberg's vastly superior standing, and this was the experience I had with many of the other prominent scientists I later met in Germany. While it surprised me at first, it had quite a simple reason: These men were so entirely devoted to their science and their work spoke so clearly for itself that there was really no room or reason for any pretense, be it in the form of grand manners or of false modesty. With Heisenberg there was the additional factor of his youth; as a professor at the age of 26 he was only about four years older,
although in the time scale of theorists this already put him something like two generations ahead of me.

As to my hopes for keeping wavepackets together by radiation damping, he only smiled and said that, if anything, it could of course only make them spread even more. Nevertheless he thought my calculations on the harmonic oscillator were a good start, and that I should go on to work them out for the general case. With the help of P.~A.~M.~Dirac's paper on radiation effects and a few more tricks, I managed to do that rather quickly, confirming Heisenberg's prediction, and it became my first published paper. It appeared in the {\it Physikalische Zeitschrift} as a precursor to the well known paper of Victor Weisskopf and Eugene Wigner on radiation damping and natural line widths.

Before the Christmas vacations, Heisenberg said that I should think about a topic for my doctor's thesis: This I did mostly while skiing in Switzerland after I had gone home. I knew the importance of Paul Ehrenfest's adiabatic theorem in the older quantum theory, and when I went back to Leipzig after New Year I proposed for my thesis its reformulation in quantum mechanics.

``Yes,'' said Heisenberg, ``one might do that, but I think you had better leave such things to the learned gentlemen of G\"ottingen.''

What he meant was the school of Born, which had the reputation of being particularly skilled in, and rather fond of, elaborate mathematical formalisms. Instead, he suggested something more
down to earth such as, for example, ferromagnetism or the conductivity of metals.

As to ferromagnetism, he thought that it had to be explained by an exchange integral between electrons, with the opposite sign from that in helium so as to favor a parallel rather than opposite orientation of their spins. He had shown before that the difference between the ortho and para states of the helium atom were due to the dependence of the exchange energy on their symmetry properties and had also recognized that the analogous phenomenon for the protons in the hydrogen molecule led to the two forms, ortho and para, of hydrogen. Well, his idea sounded so convincing that I felt there was no point of my going into it. It was obvious to me that Heisenberg already knew the essentials; indeed,.he soon wrote the paper on the subject that laid the groundwork for the modern theory of ferromagnetism. It was not until two years later that I somewhat embellished his treatment by the introduction of spinwaves.

\bigskip
\noindent
{\bf Electrons in crystals}

\smallskip
There was a greater challenge in his other suggestion, to do something more about the properties of metals. Going beyond the earlier work of Paul Drude and H.~A.~Lorentz, Wolfgang Pauli had already given a first new impetus to the field by invoking Fermi statistics to explain the temperature-independent paramagnetism of conduction electrons; Sommerfeld had gone further by discussing the consequences for the specific
heat and the relation between the thermal and the electric conductivity of metals. Both, however, had treated the conduction electrons as an ideal gas of free electrons, which didn't appear in the least plausible to me.

When I started to think about it, I felt that the main problem was to explain how the electrons could sneak by all the ions
in a metal so as to avoid a mean free path of the order of atomic distances. Such a distance was much too short to explain the observed resistances, which even demanded that the mean free path become longer and longer with decreasing temperature. But Heitler and London had already shown how electrons could jump between two atoms in a molecule to form a covalent bond, and the main difference between a molecule and a crystal was only that there were many more atoms in a periodic arrangement. To make my life easy, I began by considering wavefunctions in a one-dimensional periodic potential. By straight Fourier analysis I found to my delight that the wave differed from the plane wave of free electrons only by a periodic modulation.

This was so simple that I didn't think it could be much of a discovery, but when I showed it to Heisenberg he said right away: ``That's it!'' Well, that wasn't quite it yet, and my calculations were only completed in the summer when I wrote my thesis on``The Quantum Mechanics of Electrons in Crystal Lattices.''

I then left Leipzig to become for a year the assistant of Pauli in Zurich and to spend another year as Lorentz Fellow in Holland. It was not until the fall of 1930 that I returned to Leipzig, this time as Heisenberg's assistant, and by then the early days of quantum mechanics were really over, although many of its important consequences were yet to come—--and are still coming.

I don't think many of us realized that we had just gone through quite a unique era; we thought that this was just the way physics was normally to be done and only
wondered why clever people had not seen that earlier. Almost any problem that had been tossed around years before could now be reopened and made amenable to a consistent treatment. To be sure, there were a few minor difficulties left, such as the infinite self-energy of the electron and the question of how it could exist in the nucleus before beta decay; and nobody had yet derived the numerical value of the fine-structure constant. But we were sure that the solutions were just around the corner and that any new ideas that might be called for in the process would be easily supplied in the unlikely event that this should be necessary. Well, the last fifty years have taught us at least to be a little more modest in our expectations.

\bigskip
\noindent
{\bf Heisenberg the teacher and scientist}

\smallskip
From what I have told about the year when I had the good fortune to be Heisenberg's first student it may already be evident that he stands in the center of my memories of this most formative period in my life as a physicist. It is not only that he suggested the theme of my thesis, but I owe it to him that I caught the real spirit of research and that I dared to take the first steps in learning how to walk. If I should single out one of his great qualities as a teacher, it would be his immensely positive attitude towards any progress and the encouragement he thereby conferred.

This does not mean that one always received praise from him and that, on occasions, he could not be quite severe. Once during my thesis work I became stuck on a rather awkward difficulty and hoped that he would help me out. But
after I had explained it to him he only said: ``Now that you have analyzed the source of the trouble it can't be all that hard to see what to do about it.''

Of course, I felt rather depressed, but just to get out of it I pushed once more and in some cumbersome way finally managed indeed to get over the obstacle. It was not the mathematical method but only physical content that ever mattered to Heisenberg. As to elegance he might have agreed with Ludwig Boltzmann's opinion that it was ``best left to tailors and bootmakers.''

Besides my year as Heisenberg's student, I spent the two more years, 1930--31 and 1932--33, in Leipzig until Hitler succeeded in forming a new Germany in his own frightful image. What followed is too well known for me to dwell upon, but I cannot refrain from one sad comment on human nature. The very devotion to their work and their detachment from the dark irrational passions spreading around them caught most of even the finest German scientists unprepared for the oncoming flood. Those who did not leave were with few exceptions swept along and
were left, each in his own way, to struggle with their inner conflicts.

But my memories of Heisenberg belong to the happier time before those events. Many of them relate to entirely informal and anything-but-professional conversations on walks, in his ski hut in the Bavarian Alps or under other relaxed circumstances. These remain no less precious to me than our talks on physics, and I want to tell in conclusion about two of them that I remember most vividly.

Once I came back after dinner to my room in the Institute to finish some work. While I sat at my desk I heard Heisenberg, who was an excellent pianist, playing in his apartment under the roof of the
building. It was already late at night when he came down to my room and said he just wanted to talk a little before going to bed after he had practiced a few bars of a Schumann concerto for three hours. And then he told me that Franz Liszt, when he was already a famous pianist, found that his scales of thirds and fifths were not smooth enough. So he cancelled all engagements, and for a year practiced nothing but these scales before he started to perform again. The reason I remember this so well is that I felt that Heisenberg, without intention, had told me something important about himself. The audience of Liszt after that year must have thought it a wonder how easily he was able to play those difficult scales. But the real wonder was of course that he had had the strength and the gift of concentration to keep on perfecting them incessantly for a whole year.

Now, one of the most marvellous traits of Heisenberg was the almost infallible intuition that he showed in his approach to a problem of physics and the phenomenal way in which the solutions came to him as if out of the blue sky. I have asked myself whether that wasn't a form of the ``Liszt phenomenon,'' and for that the more admirable. Not that Heisenberg would ever have cancelled all other activity for a year to master a special technique. But we all knew the dreamy expression on his face, even in his complete attention to other matters and in his fullest enjoyment of jokes or play, which indicated that in the inner recesses of the brain he continued his all-important thoughts on physics.

There is another remark he once made that I consider even more characteristic. We were on a walk and somehow began to talk about space. I had just read Weyl's book {\it Space, Time and Matter}, and under its influence was proud to declare that space was simply the field of linear operations.

``Nonsense,'' said Heisenberg, ``space is blue and birds fly through it.''

This may sound naive, but I knew him well enough by that time to fully understand the rebuke. What he meant was that it was dangerous for a physicist to describe Nature in terms of idealized abstractions too far removed from the evidence of actual observation. In fact, it was just by avoiding this danger in the previous description of atomic phenomena that he was able to arrive at his great creation of quantum mechanics. In celebrating the fiftieth anniversary of this achievement, we are vastly indebted to the men who brought it about: not only for having provided us with a most powerful tool but also, and even more significant, for a deeper insight into our conception of reality.

\end{document}

\bigskip
\noindent
{\it This article is an adaptation of a talk given 26 April 1976 at the Washington, DC meeting of The American Physical Society.}

\end{document}
