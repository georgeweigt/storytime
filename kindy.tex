\documentclass[12pt]{article}
\usepackage{parskip}
\title{How Pearl Harbor Forced the World’s First Around-the-World Commercial Flight}
\author{Dave Kindy}
\date{2021}
\pdfinfoomitdate=1
\pdftrailerid{}
\begin{document}

\maketitle

Jack Poindexter walked briskly into the Liberty House department store on King Street in downtown Honolulu. It was Dec.~2, 1941, and palm trees swayed to the gentle rhythm of the trade winds that sunny Tuesday morning.

The chief flight radio officer on Pan Am Flight NC18602 needed a spare shirt. He had left California unexpectedly the day before as a stand-in for an ill radio man onboard the Pacific Clipper, a large flying boat --- essentially a seaplane on steroids.

Poindexter had no clean clothes for the flight, which still had to make another stop in Auckland, New Zealand, and was not scheduled to return to San Francisco until Dec.~10. He had only a few dollars in his wallet, so this extra shirt was going to have to last him until then. Little did he know it would be the only change of clothing he would have for more than a month.

The return flight from New Zealand to San Francisco via Honolulu was interrupted by the Japanese bombing of Pearl Harbor on Dec.~7, 1941 --- ``A date which will live in infamy,'' as President Franklin D.~Roosevelt termed it. That event, 80 years ago Tuesday, propelled the United States into war and forced the Pacific Clipper’s crew of 12 to make a dangerous --- and historic --- detour from their scheduled flight plan.

Instead of heading home by going east, they took the massive Boeing 314 in the opposite direction, flying blind with no charts and no support from the airline. They were shot at twice, narrowly escaped getting blown up and otherwise avoided disaster while piloting the first commercial flight to circumnavigate the globe. They flew more than 30,000 miles over vast expanses of empty oceans and remote landscapes on five continents while crossing the equator four times.

To top it off, the crew managed this feat without the aid of maps or radio contact, using only celestial navigation and an atlas obtained from a library in New Zealand.

``Flying around the world with no charts is astounding,'' said F.~Robert van der Linden, curator of Air Transportation and Special Purpose Aircraft at the Smithsonian National Air and Space Museum. ``They didn’t get lost and they had only one engine problem, which they fixed. These planes were beautiful, but they were hard to fly.''

The Boeing 314 was one of the largest aircraft of its era --- nearly as large as a Boeing 747 today. With a wingspan of 152 feet, the plane weighed 84,000 pounds when loaded with passengers and fuel, requiring the full power of its four 1,600-horsepower Wright Cyclone engines to help it break free of the water’s pull and get into the air.

Pan Am pioneered the use of flying boats in the 1930s. A lack of airfields and the need for travel to remote locations made this unique aircraft a necessary form of transportation. A long stretch of calm water was all the landing strip this behemoth needed to reach populous seaports in the United States and Europe or exotic destinations like Honolulu, Fiji and Java.

Service on board the flying boat --- essentially a glamorous luxury ocean liner in the sky --- was exceptional, with gourmet meals cooked in onboard galleys and served in dining rooms. The Boeing 314, which could carry 74 passengers, also had sleeping quarters with turndown bed service.

Tickets weren’t cheap. The cost to fly one way from San Francisco to Hong Kong in 1940 was \$760 --- nearly \$15,000 today.

``If you got on an airplane like this then, you were flying premium first class with a suite in an airliner today,'' van der Linden said. ``There were nearly as many crew members on these flights as passengers to ensure comfort and safety.''

Flight NC18602 originated in San Francisco on Dec.~1, 1941, with Capt.~Robert Ford at the helm. He was a Master of Ocean Flying Boats --- the certification pilots received once they learned the intricacies of flying such a complex aircraft. His skills would be tested to the extreme on this trip as the plane encountered all manner of obstacles, natural and man-made.

After leaving Honolulu, the Pacific Clipper headed for Canton Island, a refueling stop almost 2,000 miles southwest of Hawaii. The plane was approaching Auckland with 12 passengers on Dec.~7, 1941, when a bulletin crackled across the radio: Pearl Harbor had been attacked.

Pan Am was prepared for this eventuality. Ford had been handed a ``Top Secret'' letter just before leaving San Francisco. If war broke out, he was to follow his instructions exactly.

At the time, the Boeing 314 was considered state-of-the-art technology. The letter was explicit: If the crew could not deliver the Pacific Clipper to the American military due to attack or imminent capture, then they were to destroy the aircraft. Their fate was secondary.

As soon as the passengers exited the plane in Auckland, the crew began preparations for what was to come next. There would be no more passengers, or first-class service, onboard the Pacific Clipper from here on out. The rest of the trip would be escape and survival.

After a week in New Zealand, Ford received orders from Pan Am to fly west with his crew. His destination was New York City, which meant he would have to fly over some of the most inhospitable lands and seas in the world. Worse, Ford and his crew were on their own. They would have no support, fuel or even money from the company.

Overloaded with gas, oil, spare parts and provisions, the Pacific Clipper departed Auckland on Dec.~17 and headed for Australia. Just before departure, Ford was able to secure a \$500 advance from the Pan Am ticketing office. That would cover food and fuel for the rest of the trip.

Just before beginning the journey, the crew had stripped off most markings from the plane on orders from Pan Am, presumably to confuse Japanese patrol planes. But the move also befuddled the Royal Netherlands East Indies Army Air Force in Indonesia. Dutch fighters, already in dogfights with enemy aircraft, now looked suspiciously at the large gray flying boat trying to land in Java. Over the radio, the Pan Am crew could hear pilots wondering if they should shoot down the intruder.

Ford maintained strict radio silence, as ordered by Pan Am. He flew without any sudden course corrections and landed the flying boat in the harbor. Ford then noticed a small tender with a furiously waving boatman, who shouted that the plane had landed in a minefield. The Pan Am pilot gingerly moved his aircraft forward and avoided any contact with floating explosives.

Aviation fuel was in short supply in Java, so the crew had to pump automobile gasoline into the reserve tanks. Ford decided to take off using aviation fuel and then switch to the lower-octane gas in midflight. The engines would run hotter and could suffer damage, but he had no choice.

The Homeric odyssey continued as the plane flew on to Ceylon, now Sri Lanka. Just before landing there, Ford eased his plane below the clouds --- only to discover an enemy vessel about 300 feet below him.

``All of a sudden there it was, right in front of us, a submarine!'' Ford later recalled. ``We could see the crew running for the deck gun.'' He quickly pulled back into the clouds and avoided the incoming enemy fire.

On Christmas Eve morning, the Pacific Clipper departed Ceylon --- only to return an hour later. Just as it reached cruising altitude, a huge explosion shook the aircraft. A piston in the No.~3 engine had broken loose, causing an eruption of smoke and flame. Repairs were made, and the Boeing 314 was back in the air the next day, headed for Karachi in what is now Pakistan.

On Dec.~29, the giant aircraft lumbered over the Arabian Peninsula on its way to the Nile River in Africa. It flew over Mecca, where overflights were banned. Suddenly, the crew could see people streaming from a mosque and firing guns at the plane.

``At least they didn’t have any antiaircraft,'' the pilot recalled.

Ford eventually set down on the Nile near Khartoum. The plane was refueled and back in the air on New Year’s Day, 1942. Next stop: the Congo River in West Africa.

Departures and arrivals are two of the most difficult maneuvers on a flying boat, which requires more than a thousand yards of calm, clear water for smooth flight to begin or end. At seaports, channels are carefully maintained and checked for debris just before touchdown or liftoff.

``A flying boat doesn’t land --- it alights on water,'' van der Linden said. ``Coming down on a river is a lot harder than it sounds. You don’t know what’s in that river. When you consider the amount of fallen trees that must be in the Congo, in particular, it’s amazing. When you’re coming down at over 100 miles per hour and you hit something, you sink.''

The crew was nervous as the plane approached Leopoldville, now Kinshasa in the Democratic Republic of Congo. The river runs fast and features cataracts in several areas. Plus, the abutting jungle is thick with overgrown trees jutting into the water. Ford set the plane down safely, then made his way to the docks. As he exited, he was handed a cold beer by a member of the Pan Am ground crew, which staffed a small outpost there.

``That was one of the high points of the whole trip,'' he recalled.

One of the most difficult parts of the journey still lay ahead. The flight over the Atlantic to Brazil would be the longest stretch over open ocean. The crew flew nonstop for 20 hours, covering more than 3,500 miles to the eastern tip of the country, which was about as far as the Boeing 314 could fly without refueling.

From here, the rest of the trip was anticlimactic. The flying boat touched down briefly in the Caribbean before resuming its journey to LaGuardia Airport in New York City, where it landed on Jan.~6, 1942. At last, it was over.

All told, the Pacific Clipper had logged 209 hours in the air and traveled 31,500 miles around the globe. Even though the plane did not return to its starting point in San Francisco, historians and aviation experts were quick to call the flight the first commercial circumnavigation of the globe, since the aircraft made it back to its country of origin.

``It happened at a time when both oceans had been crossed before,'' van der Linden said. ``And it was not the first around-the-world flight. Several pilots had done it before. But those were all planned trips with maps and coordinates worked out in advance. What you have here is a commercial airliner with a commercial airliner crew doing this completely unplanned journey while under threat of being shot down.''

The Pacific Clipper’s record for the longest commercial flight by mileage still stands today.

\end{document}
